% !TEX root = review.tex

\subsection{Pure iron phase diagram}
\label{ssec:phase}

The earth has a solid inner core and liquid outer core.
The composition of the cores is still not very clear until now.
Over $90\%$ of it is made of \ce{Fe},
and opinions are divided on the rest parts.
From a cosmochemical and geochemical view,
the rest part consists of up to $10\%$ of \ce{Ni}\cite{McDonough:1995iz}.
But this fails to explain the core's
density, seismic wave velocities
and seismic anisotropy.
From a seismological view, the core contains
$2-3\%$ of light elements, like \ce{S}, \ce{O}, \ce{Si}, \ce{H} and \ce{C}, etc.
Especially the inner core,
according to seismological models, it is anisotropic, layered, and laterally heterogeneous, but the origins are not fully understood.
Besides,
at the inner-core-boundary (ICB),
the temperature must be the liquidus temperature of what the core is made of.
Thus \ce{Fe}'s melting temperature at the pressure of
the ICB will give a close estimation.
So to study iron's behavior under extreme high pressure and temperature
conditions is of great importance in understanding the earth's core.
That is to say, we need a complete phase diagram of \ce{Fe}, with
$T$ and $P$ as horizontal and vertical axes.

\begin{figure}[h]
	\centering
	\begin{minipage}[t]{.5\linewidth}
		\centering
		\includegraphics[width=\linewidth]{s1996}
		\caption{Possible high-pressure phase diagram of Fe, including established phases    bcc, fcc, bcc, and hcp as well as proposed some phases dhcp and unknown up to
			1996\cite{Soderlind:1996du}.}
		\label{fig:fepd:a}
	\end{minipage}%
	\hfil
	\begin{minipage}[t]{.5\linewidth}
		\centering
		\includegraphics[width=\linewidth]{shen1998}
		\caption{Shen \textit{et. al.} have improved their laser-heated DAC to give a
			more }
		\label{fig:fepd:b}
	\end{minipage}
\end{figure}